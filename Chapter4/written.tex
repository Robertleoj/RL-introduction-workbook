\documentclass[a4paper,11pt,reqno]{amsart}
\usepackage{amssymb}
\usepackage{graphicx}
\usepackage{tikz}
\usepackage{svg}
\usepackage{amsmath}
\usepackage{float}
\usepackage{caption}
\usepackage{geometry}
\usepackage{enumerate}
\usepackage[T1]{fontenc}
\usepackage{hyperref}
\usepackage[utf8]{inputenc}
\usepackage{listings}
\usepackage{esvect}
\usepackage{scalerel}
\usepackage{csquotes}
\usepackage{gensymb}
\usepackage{mdframed}
\geometry{left=2.5cm, right=2.5cm, bottom=3cm}
% \pagenumbering{roman}

\parskip 1ex
\parindent 0 pt

\newcounter{temp}
\newcounter{prob_counter}
\newcounter{sprob_counter}

\newenvironment{problem}
{\begin{list}{{\bf \arabic{prob_counter}}}{
      \usecounter{prob_counter}
      \addtolength{\labelsep}{.6ex}
      \addtolength{\itemsep}{4.3ex}
      \setlength{\leftmargin}{1.4em}}
      \setcounter{prob_counter}{\value{temp}}
}
{\setcounter{temp}{\value{prob_counter}}
  \end{list}
}

\newenvironment{subprob}
{
  \begin{list}{{\bf \alph{sprob_counter}}}{
      \usecounter{sprob_counter}
      \addtolength{\labelsep}{.6ex}
      \addtolength{\itemsep}{.5ex}
      \setlength{\leftmargin}{1.7em}}
}
{\end{list}}

\newenvironment{solution}{\textbf{Solution.}}{\qed}

\newcommand{\rubrik}[1]{\bigskip \begin{center}{\bf #1}\end{center} \medskip}

\newmdtheoremenv{theorem}{Theorem}
\newmdtheoremenv{corrolary}{Corrolary}
\newmdtheoremenv{definition}{Definition}
\newmdtheoremenv{remark}{Remark}
   

\usepackage{xcolor}
\definecolor{keywords}{RGB}{255,0,90}
\definecolor{comments}{RGB}{0,0,113}
\definecolor{red}{RGB}{160,0,0}
\definecolor{green}{RGB}{0,150,0}
\lstset{language=Python,
    basicstyle=\ttfamily\footnotesize,
    keywordstyle=\color{keywords},
    commentstyle=\color{comments},
    stringstyle=\color{red},
    showstringspaces=false,
    identifierstyle=\color{green},
    %procnamekeys={def,class}
    numbers=left,
    xleftmargin=2em,
    frame=single,
    framexleftmargin=2.5em
}


\lstset{literate=%
  *{0}{{{\color{green}0}}}1
    {1}{{{\color{green}1}}}1
    {2}{{{\color{green}2}}}1
    {3}{{{\color{green}3}}}1
    {4}{{{\color{green}4}}}1
    {5}{{{\color{green}5}}}1
    {6}{{{\color{green}6}}}1
    {7}{{{\color{green}7}}}1
    {8}{{{\color{green}8}}}1
    {9}{{{\color{green}9}}}1
}

\newcommand{\vect}[1]{\boldsymbol{#1}}

\begin{document}

\pagestyle{empty}
\thispagestyle{empty}

{\small{\sc\noindent
        Róbert Leó Jónsson ({\tt robertt20@ru.is}))
}}

\rubrik{Written Exercises: Chapter 4}

\section*{Exercise 4.1}
We are not using any discount, so we have
\begin{align}
    q_{\pi }(11, down) &= r + v_{\pi }(T)\\
    &= -1 + 0 = -1 \\
    q_{\pi }(7, down) &= r+ v_{\pi }(11) \\
    &= -1 + -14 = -15  
\end{align}

\section*{Exercise 4.2}
Adding the new state, we get
\begin{align}
    v_{\pi }(15) &= \sum_{a} \pi (a \mid 15)(-1 + v_{\pi }(s')\\
    &= 0.25 \cdot ( - 1  + v_{\pi }(13)) + 0.25 \cdot ( - 1 + v_{\pi }(12)) + 0.25 \cdot (-1 + v_{\pi }(14)) + 0.25 \cdot ( - 1 v_{\pi }(15)) \\
    &= 0.25(-4  + v_{\pi }(13) + v_{\pi }(12) + v_{\pi }(14) + v_{\pi }(15) \\
    &= -1 + 0.25(-20 -22 -14 + v_{\pi }(15))  \\
    &= -1 - 14 + 0.25v_{\pi }(15) \\
    \implies 0.75 v_{\pi }(15) &= -15\\
    \implies v_{\pi }(15) &= \frac{-15}{0.75} = -20
\end{align}

Now, if we change the dynamics such that moving down in state $13$ will move us to $15$, the value function will not change. This is because $v_{\pi }(13) = v_{\pi }(15)$, so $q_{\pi }(13, down) = -20$ hold regardless of whether the move will take us to $15$ or not. 

\section*{Exercise 4.3}
The equations analogous to (4.3) and (4.4) are given in Exercise 3.17. 

The update equation for $q_{\pi }(s, a)$ is given by
\begin{align}
q_{k + 1}(s, a) &= \mathbb{E}_{\pi }\left[ R_{t+1} + \gamma G_{t + 1} \mid S_{t} = s, A_{t} = a \right]  \\
&= \sum_{s', r} p(s', r \mid s, a) \left( r + \gamma \sum_{a'} \pi (a' \mid s)q_{k} (a', s) \right) 
\end{align}

\section*{Exercise 4.4}
We need to loop through the computed value function to check whether they are equal; two optimal policies will have the same value function.

\section*{Exercise 4.5}
The process is very similar - we use the Bellman equation for $q_{\pi }$ to evaluate the policy. 

To improve the policy, we again create the greedy policy $\pi '$ with respect to $q_{\pi }$. That is,

\begin{enumerate}
    \item Initialization

    $q(s, a) \in  \mathbb{R}$ and $\pi (s) \in \mathcal{A}(s)$ arbitrarily for all $s \in  \mathcal{S}$. 

    \item Policy evaluation

    Loop:
    \begin{itemize}
        \item $\Delta  \leftarrow 0$
        \item Loop for each $s \in \mathcal{S}$ and each $a \in \mathcal{A}(s)$ :
        \begin{itemize}
            \item $ q \leftarrow q(s, a)$
            \item $q(s, a) \leftarrow  \sum_{s', r}p(s', r \mid s, a)\left( r + \gamma q(s', \pi(s'))  \right)  $
            \item $\Delta \leftarrow \max(\Delta, |q - q(s, a)| $
        \end{itemize}
        \item until $\Delta < \theta $
    \end{itemize}

    \item Policy Improvement
    \begin{itemize}
        \item \emph{policy-stable} $\leftarrow true$
        \item For each $s \in  \mathcal{S}$:
        \begin{itemize}
            \item \emph{old-action} $\leftarrow \pi (s)$
            \item $\pi (s) \leftarrow \operatorname{argmax}_{a}\sum_{s', r} q(s, a)$
            \item If \emph{old-action} $\neq \pi (s)$ then \emph{policy-stable} $\leftarrow false$. 
        \end{itemize}
        \item If \emph{policy-stable}, then stop and return $q \approx q_{*}$ and $\pi \approx \pi _{*}$, else go to 2.
    \end{itemize}
\end{enumerate}


\section*{Exercise 4.6}
We assume that there would be a $1 - \epsilon + \frac{\epsilon}{|\mathcal{A}(s)|}$ probability on the optimal action, and the remaining $\epsilon - \frac{\epsilon}{|\mathcal{A}(s)|}$ probability would be evenly spread over the remaining actions. 

Now, assume that we store this action in $A(s)$. Then,
 \begin{align}
\pi_A (a \mid s) = 
\begin{cases}
    1 - \epsilon+\frac{\epsilon}{|\mathcal{A}(s)|} & \text{ if } a = A(s)\\
    \epsilon - \frac{\epsilon}{|\mathcal{A}(s)|} & \text{ otherwise } 
\end{cases}
\end{align}
Then, step 3 would be
\begin{itemize}
    \item \emph{policy-stable} $\leftarrow true$
    \item For each $s \in \mathcal{S}$ :
    \begin{itemize}
        \item \emph{old-action} $\leftarrow A(s)$
        \item $A(s) \leftarrow  \operatorname{argmax}_{a'}\sum_{a}\pi _a'(a \mid s) \sum_{s', r} p(s', r \mid s, a)\left[ r + \gamma V(s') \right]  $
        \item $\pi(s) \leftarrow \pi_A(s)$
        \item If \emph{old-action} $\neq A(s)$ then \emph{policy-stable} $\leftarrow false$
    \end{itemize}
    \item $\cdots $
\end{itemize}

For step two, the value update would simply be
\begin{align}
V(s) \leftarrow \sum_{a}\pi_A(s)(a \mid s) \sum_{s', r} p(s', r \mid s, a') \left[ r + \gamma V(s') \right]  
\end{align}
where the rest of the step remains the same.

Step one would consist of letting $V(s) \in  \mathbb{R}$ arbitrarily, and setting $A(s)$ for $s \in  \mathcal{S}$ arbitrarily, with $\pi (a \mid s) = \pi _{A(s)}(a \mid s)$ as defined above.







\end{document}




















